\documentclass[10pt,a4paper]{article}
\usepackage[utf8]{inputenc}
\usepackage{amsmath}
\usepackage{amsfonts}
\usepackage{amssymb}
\usepackage{tcolorbox}
\usepackage{booktabs}
\usepackage{graphicx} % for pdf, bitmapped graphics files
	\addtolength{\oddsidemargin}{-.875in}
	\addtolength{\evensidemargin}{-.875in}
	\addtolength{\textwidth}{1.75in}

	\addtolength{\topmargin}{-.875in}
	\addtolength{\textheight}{1.75in}
\newcommand\scalemath[2]{\scalebox{#1}{\mbox{\ensuremath{\displaystyle #2}}}}


\usepackage{listings}
\usepackage{color} %red, green, blue, yellow, cyan, magenta, black, white
\definecolor{mygreen}{RGB}{28,172,0} % color values Red, Green, Blue
\definecolor{mylilas}{RGB}{170,55,241}

\lstset{language=Matlab,%
    %basicstyle=\color{red},
    breaklines=true,%
    morekeywords={matlab2tikz},
    keywordstyle=\color{blue},%
    morekeywords=[2]{1}, keywordstyle=[2]{\color{black}},
    identifierstyle=\color{black},%
    stringstyle=\color{mylilas},
    commentstyle=\color{mygreen},%
    showstringspaces=false,%without this there will be a symbol in the places where there is a space
    numbers=left,%
    numberstyle={\tiny \color{black}},% size of the numbers
    numbersep=9pt, % this defines how far the numbers are from the text
    emph=[1]{for,end,break},emphstyle=[1]\color{red}, %some words to emphasise
    %emph=[2]{word1,word2}, emphstyle=[2]{style},    
}


\author{Mason Smith}
\begin{document}

% Title and header
\begin{center}
\Large{ \bf CLASS\# CLASSNAME - Assignment \#} \\ \vspace{3mm}
\large{By: Mason Smith }
\large{Submitted: 11/6/20 }
\end{center}
\section{Problem 1. READ DUAN}
\section{Problem 2. READ LECTURES}

\section{Problem 3.}
\subsection{Problem 3.a}

\begin{tcolorbox}
\textbf{SOLUTION:}\\
$$|y(t)|=\big|\int\limits_{-\infty}^{\infty} h(t-s)u(s) ds\big|$$
By the triangle inequality:
$$|y(t)|\leq \int\limits_{-\infty}^{\infty} \big|h(t-s)u(s)\big| ds
= \int\limits_{-\infty}^{\infty} |h(t-s)||u(s)| ds
= \int\limits_{-\infty}^{\infty} |h(t-s)||u(s)| ds$$
Let $||u(s)||_\infty$ be the maximum value of the of $u(s)$. Since this term is fixed, we can treat it as a constant and separate it from the integral.
$$|y(t)|\leq \int\limits_{-\infty}^{\infty} |h(t-s)|||u(s)||_\infty ds
= ||u(s)||_\infty \int\limits_{-\infty}^{\infty} |h(t-s)|ds$$
where t becomes negligible when integrating over $(-\infty,\infty)$ in the s-domain. Therefore:
$$|y(t)|\leq ||u(s)||_\infty \int\limits_{-\infty}^{\infty} |h(s)|ds$$
which the $\mathcal{L}^1$ norm is recovered:
$$|y(t)|\leq ||u(s)||_\infty ||h(s)||_1$$
Where $||h(s)||_1$ is bounded given $h\in L^1(\mathbb{R})$ and $||u(s)||_\infty$ is bounded under the assumption that we cannot produce an input signal with infinite energy.  Applying these two conclusions we can  infer that $|y(t)|$ is also bounded:
$$|y(t)| \leq ||u(s)||_\infty ||h(s)||_1<\infty$$
and conclude that for $\gamma=||h(s)||_1$ and $\beta=0$
$$||y(t)||_\infty= \sup\limits_{s\in[0,\infty)}|y(t)| \leq \gamma ||u(s)||_\infty+\beta$$
\end{tcolorbox}

\subsection{Problem 3.b BONUS}
Problem 1.b BONUS
\begin{tcolorbox}
\textbf{SOLUTION:}\\
INCOMPLETE
\end{tcolorbox}

\section{Problem 4.}
\subsection{Problem 4.a}
\begin{tcolorbox}
\textbf{SOLUTION:}\\
Given:
$$||f||_{H_m^p(\Omega)}=\big( \int\limits_\Omega \sum\limits_{k=0}^{m}|f^{(k)}(s)|^p ds\big)^{\frac{1}{p}}$$
We can expand the summation for $m+1$:
$$||f||_{H_{m+1}^p(\Omega)}
=\big( \int\limits_\Omega \sum\limits_{k=0}^{m+1}|f^{(k)}(s)|^p ds \big)^{\frac{1}{p}}$$

$$=\big( \int\limits_\Omega \sum\limits_{k=0}^{m}|f^{(k)}(s)|^p ds 
+\int\limits_\Omega |f^{(m+1)}(s)|^p ds \big)^{\frac{1}{p}}$$
Where fore any value of p and any domain $\Omega$, $\int\limits_\Omega |f^{(m+1)}(s)|^p ds \big)^{\frac{1}{p}}> 0$.  This result returns $||f||_{H_{m}^p(\Omega)}$ in addition to a positive term allowing the following inequalities to be satisfied:
$$\big( \int\limits_\Omega \sum\limits_{k=0}^{m}|f^{(k)}(s)|^p ds \big)^{\frac{1}{p}}
\leq \big( \int\limits_\Omega \sum\limits_{k=0}^{m}|f^{(k)}(s)|^p ds 
+\int\limits_\Omega |f^{(m+1)}(s)|^p ds \big)^{\frac{1}{p}}$$

$$\int\limits_\Omega \sum\limits_{k=0}^{m}|f^{(k)}(s)|^p ds 
\leq \int\limits_\Omega \sum\limits_{k=0}^{m}|f^{(k)}(s)|^p ds 
+\int\limits_\Omega |f^{(m+1)}(s)|^p ds$$

$$0\leq \int\limits_\Omega |f^{(m+1)}(s)|^p ds$$
where the term on the right has already been shown to be a non-negative term thereby satisfying the inequality and proving that
$$||f||_{H_m^p(\Omega)} \leq ||f||_{H_{m+1}^p(\Omega)}$$
\end{tcolorbox}

\subsection{Problem 4.b}
\begin{tcolorbox}
\textbf{SOLUTION:}\\
Given:
\begin{center}
$p=1$,$f(s)=e^{-s}$ and $\Omega = (0, inf)$\\
\end{center}
$$f^{(k)}(s)=\frac{d^k f(s)}{ds^k}=\frac{d^k}{ds^k}e^{-s}$$
$$||f||_{H_{m}^1(0,\infty)}
=(\int\limits_0^\infty \sum\limits_{k=0}^{m}|\frac{d^k}{ds^k}e^{-s}| ds)^{\frac{1}{p}}$$
for m=0:
$$||f||_{H_{0}^1(0,\infty)}
=\int\limits_0^\infty \sum\limits_{k=0}^{m=0}|\frac{d^k}{ds^k}e^{-s}| ds
=\int\limits_0^\infty|\frac{d^0}{ds^0}e^{-s}| ds$$

$$||f||_{H_{0}^1(0,\infty)}
=\int\limits_0^\infty|e^{-s}| ds=\int\limits_0^\infty e^{-s}= -e^{-s} \big|_{0}^\infty$$
$$= -e^{-\infty} - ( -e^{-\infty}) = 0+1$$
$$\therefore ||f||_{H_{0}^1(0,\infty)} = 1$$


for m=1:
$$||f||_{H_{1}^1(0,\infty)}
=\int\limits_0^\infty \sum\limits_{k=0}^{m=1}|\frac{d^k}{ds^k}e^{-s}| ds$$

$$||f||_{H_{1}^1(0,\infty)}
=\int\limits_0^\infty|\frac{d^1}{ds^1}e^{-s}| ds
+||f||_{H_{0}^1(0,\infty)}$$

$$||f||_{H_{1}^1(0,\infty)}
=\int\limits_0^\infty|-e^{-s}|ds+1 
=\int\limits_0^\infty e^{-s}ds+1=1+1$$
$$\therefore ||f||_{H_{1}^1(0,\infty)}=2$$

for m=2:
$$||f||_{H_{2}^1(0,\infty)}
=\int\limits_0^\infty \sum\limits_{k=0}^{m=2}|\frac{d^k}{ds^k}e^{-s}| ds$$

$$||f||_{H_{2}^1(0,\infty)}
=\int\limits_0^\infty|\frac{d^2}{ds^2}e^{-s}| ds
+||f||_{H_{1}^1(0,\infty)}+||f||_{H_{0}^1(0,\infty)}$$

$$||f||_{H_{2}^1(0,\infty)}
=\int\limits_0^\infty|e^{-s}|ds
=\int\limits_0^\infty e^{-s}ds+1+1=1+1+1$$

$$\therefore ||f||_{H_{2}^1(0,\infty)}=3$$

\end{tcolorbox}

\subsection{Problem 4.c}
\begin{tcolorbox}
\textbf{SOLUTION:}\\
This pattern leads to an observation that 
\begin{center}
$||f||_{H_{m}^1(\Omega)}=m+1$   or   $||f||_{H_{m}^1(\Omega)}=\sum\limits_{k=0}^m 1$
\end{center}

for the function $f(s) = e^{-s}$.  This appears to be caused by the derivative of f(s) to only invert its sign with ever sequential k in $f^{(k)}(s)=\frac{d^k f}{ds^k}$.  This changing of the $sign(f^{(k)})$ is then made inconsequential since the next operator is the absolute value of this result and leads to the same integration for every value of k which.  This allows us to simplify the integral for all values of $\frac{d^k}{ds^k}e^{-s}$:
$$\int\limits_{0}^\infty|\frac{d^k}{ds^k}e^{-s}|ds=\int\limits_{0}^\infty e^{-s}ds = 1$$

\end{tcolorbox}

\section{Problem 5}
\begin{tcolorbox}
\textbf{SOLUTION:}\\
\textbf{(i)Define Variables:} \\

\textbf{(i)Define Equations for Regulated Outputs in terms of Exogenous/Actuator Inputs:} \\
$z1= r-P_0(n_{proc}+u)$\\
$z2=u$\\

\textbf{(ii)Define Equations for Sensed Outputs in terms of Exogenous/Actuator Inputs:} \\
$y_1=r$\\
$y_2 = q+n_{sensor}$\\

\textbf{(iii)Construct Matrices for Regulated/Sensed Outputs:} \\
$z=\begin{bmatrix}
e\\
u
\end{bmatrix}=\begin{bmatrix}
r-P_0(n_{proc}+u)
\\u
\end{bmatrix}=\begin{bmatrix}
\omega_1-P_0(\omega_2+u) \\
 u
\end{bmatrix}$

$y=\begin{bmatrix}
r\\
q+n_{sensor}
\end{bmatrix}=\begin{bmatrix}
r\\
P_0(n_{proc}+u)+n_{sensor}
\end{bmatrix}=\begin{bmatrix}
\omega_1 \\
P_0(\omega_2+u)+\omega_3
\end{bmatrix}$\\

\textbf{(iiv)Construct Aggregate Plant:} \\
$P=\begin{bmatrix}
z_1\\z_2\\y_1\\y_2
\end{bmatrix}=\begin{bmatrix}
I& -P_0 & 0& |& -P_0\\
0& 0& 0& |& I\\
\hline
I& 0& 0& |& 0\\
0& P_0& I& |& P_0
\end{bmatrix}\begin{bmatrix}
\omega_1\\\omega_2\\\omega_3\\u
\end{bmatrix} =\begin{bmatrix}
P_{11}& |& P_{12}\\
\hline
P_{21}& |&P_{22}
\end{bmatrix}\begin{bmatrix}
\omega\\u
\end{bmatrix}$\\

\textbf{(iv)Construct State Space Representation:} \\
where $\omega_2 +u$ is our input signal\\
$$P=\begin{bmatrix}
P_{11}& |& P_{12}\\
\hline
P_{21}& |&P_{22}
\end{bmatrix}\begin{bmatrix}
\omega\\u
\end{bmatrix}$$
$$\dot{x}=Ax+B(\omega_2 +u)$$
$$y=Cx+D(\omega_2 +u)$$\\
\textbf{(v)Construct the 9-matrix representation}
$$P=\begin{bmatrix}
A& |& 0& B& 0& B&\\
\hline
C& |& I& -D& 0& -D\\ 
0& |& 0& 0& 0& I\\
0& |& I& 0& 0& 0\\
C& |& 0& D& I& D

\end{bmatrix}
$$

\end{tcolorbox}

\section{Problem 6}
\subsection{Problem 6.a}
\begin{tcolorbox}
\textbf{SOLUTION:}\\
For the regulator framework 

\textbf{(i)Define Equations for Regulated Outputs in terms of Exogenous/Actuator Inputs:} \\
$z1= P_0(\omega_1+u)$\\
$z2=u$\\

\textbf{(ii)Define Equations for Sensed Outputs in terms of Exogenous/Actuator Inputs:} \\
$y=P_0(\omega_1+u)+\omega_2$\\

\textbf{(iii)Construct Matrices for Regulated/Sensed Outputs:} \\
$$\begin{bmatrix}
z_1\\z_2
\end{bmatrix}=\begin{bmatrix}
P_0(\omega_1+u)\\u
\end{bmatrix}$$

$$\begin{bmatrix}
y
\end{bmatrix}=\begin{bmatrix}
P_0(\omega_1+u)+\omega_2
\end{bmatrix}$$

\textbf{(iiv)Construct Aggregate Plant:} \\
$$\begin{bmatrix}
z_1\\z_2\\y
\end{bmatrix}=\begin{bmatrix}
P_0& 0& |& P_0\\
0&	0&	|& I\\
\hline
P_0& I& |&P_0\\
\end{bmatrix}\begin{bmatrix}
w_1\\w_2\\w_3
\end{bmatrix}=\begin{bmatrix}
P_{11}& |& P_{12}\\
\hline
P_{21}& |&P_{22}
\end{bmatrix}\begin{bmatrix}
\omega\\u
\end{bmatrix}$$\\

\textbf{(iv)Construct State Space Representation:} \\
where $\omega_1+u$ is our input signal and the state space representation takes the form:
$$P=\begin{bmatrix}
P_{11}& |& P_{12}\\
\hline
P_{21}& |&P_{22}
\end{bmatrix}$$
$$\dot{x}=Ax+B(\omega_1 +u)$$
$$y=Cx+D(\omega_1 +u)$$

\textbf{(v)Construct the 9-matrix representation}
We can the construct our reconfigured plant:
$$P=\begin{bmatrix}
A& |& B& 0& B\\
\hline
C& |& D& 0& D\\
0& |& 0& 0& I\\
C& |& D& I& D
\end{bmatrix}
$$

\textbf{(vi)Matlab Result}
Using the values provided for A,B,C,D the full 9-matrix representation for P can be calculated using matlab:
$$P=\left(\begin{array}{ccccccccccccccc} 
-1 & 1 & 0 & 1 & 0 & 1 & 0 & -1 & -1 & 0 & 0 & 0 & 0 & -1 & -1\\ 
-1 & -2 & -1 & 0 & 0 & 1 & 0 & 0 & 0 & 0 & 0 & 0 & 0 & 0 & 0\\
 1 & 0 & -2 & -1 & 1 & 1 & -1 & 1 & 1 & 0 & 0 & 0 & -1 & 1 & 1\\ 
 -1 & 1 & -1 & -2 & 0 & 0 & -1 & 0 & 0 & 0 & 0 & 0 & -1 & 0 & 0\\
  -1 & -1 & 1 & 1 & -2 & -1 & 0 & 0 & 1 & 0 & 0 & 0 & 0 & 0 & 1\\
   0 & -1 & 0 & 0 & -1 & -3 & -1 & 1 & 1 & 0 & 0 & 0 & -1 & 1 & 1\\ 
   0 & 1 & 0 & -1 & -1 & -1 & 0 & 0 & 0 & 0 & 0 & 0 & 0 & 0 & 0\\ 
   0 & 0 & 0 & -1 & 0 & 0 & 0 & 0 & 0 & 0 & 0 & 0 & 0 & 0 & 0\\ 
   1 & 0 & 0 & 0 & -1 & 0 & 0 & 0 & 0 & 0 & 0 & 0 & 0 & 0 & 0\\ 
   0 & 0 & 0 & 0 & 0 & 0 & 0 & 0 & 0 & 0 & 0 & 0 & 1 & 0 & 0\\ 
   0 & 0 & 0 & 0 & 0 & 0 & 0 & 0 & 0 & 0 & 0 & 0 & 0 & 1 & 0\\ 
   0 & 0 & 0 & 0 & 0 & 0 & 0 & 0 & 0 & 0 & 0 & 0 & 0 & 0 & 1\\ 
   0 & 1 & 0 & -1 & -1 & -1 & 0 & 0 & 0 & 1 & 0 & 0 & 0 & 0 & 0\\ 
   0 & 0 & 0 & -1 & 0 & 0 & 0 & 0 & 0 & 0 & 1 & 0 & 0 & 0 & 0\\ 
   1 & 0 & 0 & 0 & -1 & 0 & 0 & 0 & 0 & 0 & 0 & 1 & 0 & 0 & 0 \end{array}\right)
$$

\end{tcolorbox}
\subsection{Problem 6.b}
\begin{tcolorbox}
\textbf{SOLUTION:}\\
\textbf{(i)LMI to solve $H_\infty$ optimal state-feedback problem}\\
The following are equivalent.

1) There exists a $F$ such that $||\underline{S}(P,K(0,0,0,F)||_{H_{\infty}}\leq \gamma$\\

2) There exists $Y>0$ and $Z$ such that\\

$\begin{bmatrix} YA^T+AY+Z^TB_2^{T}+B_2Z & B_1 & YC^{T}_1 + Z^{T} D^{T}_{12} \\ B1^{T}_1 & -\gamma I & D^{T}_{11} \\ C_1 Y + D_{12} Z & D_{11} & -\gamma I \end{bmatrix}<0
$\\

Then $F=ZY^{-1}$\\

The lower linear fractional transformation (LFT) is used to implement a controller $K$ into the system. The lower LFT is denoted as $\underline{S}(P,K)$ and is formed by $\underline{S}(P,K)=P_{11}+P_{12}(I-KP_{22})^{-1}KP_{21}$ with $\begin{bmatrix} z\\y \end{bmatrix}=\begin{bmatrix} P_{11} & P_{12} \\ P_{21} & P_{22} \end{bmatrix} \begin{bmatrix} w\\u \end{bmatrix}$\\

For full-state feedback we consider a controller of the form $u(t)=Fx(t)$. This is a special case where $y(t)=x(t)$ and results in a controller of the form 
$ K= \begin{bmatrix} 0 & 0 \\ 0 & F \end{bmatrix} $.

\textbf{(ii)Matlab Result}
$$F=\begin{bmatrix}

    0.6817&   -1.2214&    0.7896&    0.8756&    1.6454&    0.6413\\
    0.5895&    1.3320&   -0.7950&   -0.2227&   -1.4536&    0.2842\\
    1.1663&    3.1650&   -2.7014&   -1.0259&   -2.7165&    1.4060\\
    \end{bmatrix}$$
$||\underline{S}(P,K(0,0,0,F)||_{H_{\infty}}< 0.9476$  \\  
$H_\infty$ closed-loop gain = 0.9476\\

\end{tcolorbox}
\subsection{Problem 6.c}
\begin{tcolorbox}
\textbf{SOLUTION:}\\
\textbf{(i)Does it exist?}\\
The $H_2$ closed loop gain requires $D_cl=0$ and since this is true for the controller designed in part b, the $H_2$ closed loop gain exists.\\

\textbf{(ii)Define the close loop representation for controller}
\begin{align*}
A_{cl} = A+B_2*F\\
B_{cl} = B_1\\
C_{cl}= C_1+D_12*F\\
D_{cl} = D_11
\end{align*}

\textbf{(iii)LMI for $H_2$ closed loop gain}\\
Assuming that $\hat{P}(s)=C(sI-A)^{-1}B$, this means that the following are equivalent:

$ 1)\quad A\text{ is Hurwitz and }||\hat{P}||^2_{H_2}<\gamma$\\

2)\\
$\begin{matrix}
trace(C_{cl}XC_{cl}^T )&<\gamma\\
A_{cl}X+XA_{cl}^T +B_{cl}B_{cl}^T &<0\\
X_{cl}&>0
\end{matrix}$\\

\textbf{(iiv)Matlab Result}\\
$$X=\left(\begin{array}{cccccc} 0.2447 & -0.0490 & -0.0923 & 0.0281 & -0.0720 & -0.0909\\ -0.0490 & 0.0273 & -0.0023 & 0.0204 & 0.0169 & 0.0033\\ -0.0923 & -0.0023 & 0.2416 & 0.0558 & 0.0082 & 0.2136\\ 0.0281 & 0.0204 & 0.0558 & 0.1277 & -0.0069 & 0.0658\\ -0.0720 & 0.0169 & 0.0082 & -0.0069 & 0.1077 & -0.0063\\ -0.0909 & 0.0033 & 0.2136 & 0.0658 & -0.0063 & 0.2076 \end{array}\right)
$$
$||\hat{P}||_{H_{2}} < 2.1699$  \\  
$H_2$ gain=2.1699

\end{tcolorbox}


\subsection{Problem 6.d}
\begin{tcolorbox}
\textbf{SOLUTION:}\\
\textbf{(ii)LMI for $H_2$-optimal state feedback control}\\
The following are equivalent.\\
1) There exists a $K$ such that $||S(K,P)||_{H_{2}}<\gamma$\\
2) There exists $X>0$, $Z$ and $W$ such that\\

$\begin{bmatrix} A & B_2 \end{bmatrix} \begin{bmatrix} 
X \\ Z \end{bmatrix}+ \begin{bmatrix} 
X & Z^T \end{bmatrix} \begin{bmatrix} A^T \\ B_2^T \end{bmatrix}+ B_1B_1^T <0 
$\\

$\begin{bmatrix} 
X & *^T \\ C_1X+D_{12}Z  & W \end{bmatrix} > 0 $\\
$trace(W)<\gamma^2$\\
where $K=ZX^{-1}$\\

\textbf{(iii)Matlab Result}

$$K=\left(\begin{array}{cccccc} 0.1348 & -0.3049 & 0.0313 & 0.5380 & 0.2327 & 0.2856\\ 0.3234 & 0.2382 & -0.0867 & -0.1179 & -0.3702 & 0.0225\\ 0.5747 & 0.4936 & -0.2082 & -0.2317 & -0.7455 & 0.0251 \end{array}\right)
$$
$||S(K,P)||_{H_{2}}<1.6572$
$H_2$ gain = 1.6572
\end{tcolorbox}
\subsection{Problem 6.e}
\begin{tcolorbox}
\textbf{SOLUTION:}\\
\textbf{(i) LMI for $H_\infty$ gain }\\
\textit{Note: used dialated KYP lemma}\\

\textbf{(iii)Matlab Result}
$$X=\left(\begin{array}{cccccc} 1.2639 & 0.0550 & 0.2727 & -0.0282 & 0.0170 & 0.2706\\ 0.0550 & 1.0329 & -0.2907 & -0.0647 & -0.5094 & 0.1786\\ 0.2727 & -0.2907 & 0.7912 & -0.0994 & 0.3676 & -0.0561\\ -0.0282 & -0.0647 & -0.0994 & 0.8802 & 0.1414 & 0.1008\\ 0.0170 & -0.5094 & 0.3676 & 0.1414 & 0.9138 & -0.1184\\ 0.2706 & 0.1786 & -0.0561 & 0.1008 & -0.1184 & 0.6175 \end{array}\right)
$$
$H_\infty$ gain of closed loop system = 1.0813

\end{tcolorbox}


\section{Problem 7}
\subsection{Problem 7.a}
\begin{tcolorbox}
\textbf{SOLUTION:}\\
The following are equivalent:
There exists a $\hat{K}=\begin{bmatrix} A_K & B_K \\ C_K & D_K \end{bmatrix}$ such that $||S(K,P)||_{H_{\infty}}<\gamma$\\
There exists $X_1$, $Y_1$, $Z$, $A_n$, $B_n$, $C_n$, $D_n<$ such that

$$\begin{bmatrix} 
X_{1}      & I \\ 
I          & Y_{1}
\end{bmatrix} > 0 
$$

$$\begin{bmatrix} 
AY_{1}+Y_{1}A^{\text{T}}+B_{2}C_{n}+C_{n}B_{2}^{\text{T}}  & *^{\text{T}} & *^{\text{T}} & *^{\text{T}} \\ 
A^{\text{T}}+A_{n}+(B_{2}D_{n}C_{2})^{\text{T}}  & X_{1}A+A^{\text{T}}+B_{n}C_{2}+C_{2}^{\text{T}}B_{n}^{\text{T}} & *^{\text{T}} & *^{\text{T}} \\ 
(B_{1}+B_{2}D_{n}D_{21})^{\text{T}} & (X_{1}B_{1} + B_{n}D_{21})^{\text{T}} & -\gamma I & *^{\text{T}}\\ 
C_{1}Y_{1}+D_{12}C_{n} & C_{1}+D_{12}D_{n}C_{2} & D_{11}+D_{12}D_{n}D_{21} & -\gamma I\\
\end{bmatrix} < 0
$$

\textbf{Matlab Result:}
$H_\infty gain = 1.1072$
\end{tcolorbox}

\subsection{Problem 7.b}
\begin{tcolorbox}
\textbf{SOLUTION:}\\
(i) Construct the corresponding controller\\
The above LMI determines the the upper bound $\gamma$ on the $H_{\infty}$ norm. In addition to this the controller $\hat{K}(A_K,B_K,C_K,D_K)$ can also be recovered.\\
$ D_K=(I+D_{K2}D_{22})^{-1}D_{K2}$\\
$ B_K=B_{K2}(I+D_{22}D_K)$\\
$C_K=(I+D_{K}D_{22})C_{K2}$\\
$A_K=A_{K2}-B_K(I+D_{22}D_K)^{-1}D_{22}C_K$\\

where,
$\begin{bmatrix} A_{K2} & B_{K2} \\ C_{K2} & D_{K2} \end{bmatrix}= {\begin{bmatrix} X_{2} & X_{1}B_2 \\ 0 & I \end{bmatrix}}^{-1} \left[\begin{bmatrix} A_n & B_n \\ C_n & D_n \end{bmatrix}-\begin{bmatrix} X_1AY_1 & 0 \\ 0 & 0 \end{bmatrix}\right] {\begin{bmatrix} Y_2^T & 0 \\ C_2Y_1 & I \end{bmatrix}}^{-1}
$\\

for any full-rank $X_2$ and $Y_2$ such that\\
$$\begin{bmatrix} X_1 & X_2 \\ X_2^T & X_3 \end{bmatrix}= {\begin{bmatrix} Y_1 & Y_2 B_2 \\ Y_2^T & Y_3\end{bmatrix}}^{-1}
$$

$\therefore$ we can set $Y_2 = I$ and $X_2 = I-X_1*Y_1$ and perform the above operations in inverse order to on our feasible solutions to our optimization problem to find $A_K,B_K,C_K,D_K$ and construct the controller:
$$K=\begin{bmatrix}
A_K& B_k\\
C_K& D_k
\end{bmatrix}
$$


\textbf{Result}
$$K=\left(
\scalemath{0.6}{
\begin{array}{ccccccccc} 8.2264e+04 & 1.0279e+04 & -2.6134e+05 & -5.1350e+04 & 2.9243e+05 & 4.1978e+04 & 2.1486e+04 & 290.9353 & 2.9722e+04\\ 7.4234e+04 & 9.2721e+03 & -2.3584e+05 & -4.6341e+04 & 2.6390e+05 & 3.7892e+04 & 1.9390e+04 & 262.7264 & 2.6822e+04\\ -5.0414e+04 & -6.2962e+03 & 1.6016e+05 & 3.1469e+04 & -1.7921e+05 & -2.5725e+04 & -1.3167e+04 & -178.3610 & -1.8214e+04\\ -1.9179e+04 & -2.3969e+03 & 6.0933e+04 & 1.1971e+04 & -6.8187e+04 & -9.7979e+03 & -5.0106e+03 & -68.2194 & -6.9307e+03\\ -1.0883e+05 & -1.3594e+04 & 3.4572e+05 & 6.7930e+04 & -3.8685e+05 & -5.5532e+04 & -2.8423e+04 & -385.0597 & -3.9318e+04\\ 2.7914e+04 & 3.4872e+03 & -8.8679e+04 & -1.7424e+04 & 9.9226e+04 & 1.4239e+04 & 7.2903e+03 & 98.6305 & 1.0085e+04\\ 3.8985 & -0.9459 & -5.2285 & 0.6166 & 4.1976 & -3.3055 & 0.0089 & 0.0028 & -0.0038\\ -4.0558 & 1.2173 & 7.2609 & 0.6405 & -5.5473 & 4.8033 & 0.0028 & -0.0049 & -0.0018\\ -2.5704 & 0.8638 & 4.8891 & 0.4711 & -5.1088 & 3.0666 & -0.0038 & -0.0018 & -0.0079 
\end{array}
}
\right)
$$

\textbf{(ii) show that it achieves the predicted closed-loop Hinf gain.}\\

To construct the SS representation of the closed loop sys:\\
$Acl =\begin{bmatrix}
A& 0\\
0& Ak 
\end{bmatrix} +\begin{bmatrix}
B2& 0\\0 &Bk
\end{bmatrix}\begin{bmatrix}
I& -Dk\\ -D22& I
\end{bmatrix}^{-1}\begin{bmatrix}
0& Ck\\ C2&0
\end{bmatrix}
$\\

$Bcl=\begin{bmatrix}
B1+ B2*Dk*Q*D21\\
      Bk*Q*D21
\end{bmatrix}
$\\


$Ccl =\begin{bmatrix}
C1& 0\end{bmatrix}+\begin{bmatrix}
D12& 0
\end{bmatrix}\begin{bmatrix}
I& -Dk\\-D22, I
\end{bmatrix}^{-1} \begin{bmatrix}
0& Ck\\ C2&0;
\end{bmatrix}$\\

$Dcl = \begin{bmatrix}
D11+D12*Dk*Q*D21
\end{bmatrix} $\\

where the matlab system analysis commands con be used:\\
$sys = ss(Acl, Bcl, Ccl, Dcl);$\\
$Hinf_{ctrl} = norm(sys,inf);$\\
\textbf{Result}\\
(Returned from Matlab SS System Analysis) $H_\infty$ gain = 1.1070\\

Therefore the controller approx achieves the predicted gain since the following terms are approximately equal:\\
   $H_\infty(LMI)$ = $H_\infty(controller)$; 1.1072=1.107
   
\end{tcolorbox}
\subsection{Problem 7.c}
\begin{tcolorbox}
\textbf{SOLUTION:}\\
\textbf{(i) Use an LMI to formulate and solve the H2-optimal ouput-feedback problem.} 

1) There exists a $\hat{K}=\begin{bmatrix} A_K & B_K \\ C_K & D_K \end{bmatrix}$ such that $||S(K,P)||_{H_2}<\gamma$\\

2) There exists $X_1$, $Y_1$, $Z$, $A_n$, $B_n$, $C_n$, $D_n$ such that\\

$\begin{bmatrix} 
AY_{1}+Y_{1}A^{\text{T}}+B_{2}C_{n}+C_{n}B_{2}^{\text{T}}  & *^{\text{T}} & *^{\text{T}} \\ 
A^{\text{T}}+A_{n}+(B_{2}D_{n}C_{2})^{\text{T}}  & X_{1}A+A^{\text{T}}+B_{n}C_{2}+C_{2}^{\text{T}}B_{n}^{\text{T}} & *^{\text{T}}\\ 
(B_{1}+B_{2}D_{n}D_{21})^{\text{T}} & (X_{1}B_{1} + B_{n}D_{21})^{\text{T}} & -\gamma I
\end{bmatrix} < 0$\\

$\begin{bmatrix} 
Y_{1}   & *^T & *^T \\ 
I  & X_{1} & *^T\\  
C_{1}Y_{1}+D_{12}C_{n} & C_{1}+D_{12}D_{n}C_{2} & Z
\end{bmatrix} > 0 $\\

$D_{11} + D_{12}D_{n}D_{21} = 0 $\\

$ \text{trace}(Z) < \gamma^{2} $\\

\textbf{(ii) What H2 gain diid you find?}\\ 
(Returned from LMI) $H_2$ gain =  1.9340

\end{tcolorbox}
\subsection{Problem 7.d}
\begin{tcolorbox}
\textbf{SOLUTION:}\\
(i) Construct the corresponding controller
$$K=\left(
\scalemath{0.7}{
\begin{array}{ccccccccc} 7.8330 & 0.3149 & 18.4869 & 16.8449 & 2.3027 & 17.8954 & 32.3854 & 28.3374 & -20.5784\\ -10.3488 & 2.5522 & 45.8789 & 19.9332 & 20.5579 & 34.7307 & 52.2954 & 28.4836 & -11.9234\\ -35.6560 & 4.6924 & 16.7731 & -11.1883 & 26.7231 & 0.0492 & -16.3393 & -30.7800 & 45.8769\\ -19.3020 & 2.0994 & -17.5058 & -19.3538 & 5.5660 & -21.0804 & -43.6287 & -40.4138 & 38.7429\\ 38.2431 & -4.1367 & -5.2653 & 20.1470 & -24.8815 & 11.2499 & 31.8608 & 45.0072 & -57.0023\\ 45.4199 & -5.7606 & -25.4699 & 13.0189 & -36.7500 & -1.0806 & 13.7220 & 35.5777 & -56.7713\\ -0.0102 & 0.0020 & 0.0423 & 0.0145 & 0.0270 & 0.0219 & -1.0143e-11 & -3.2897e-12 & -2.4201e-12\\ 0.0650 & -0.0059 & -0.0759 & -0.0063 & -0.0908 & -0.0049 & 9.1441e-12 & 1.1925e-13 & 1.1959e-11\\ 0.1259 & -0.010 & -0.1556 & -0.0118 & -0.1793 & -0.0130 & 1.8773e-11 & 4.0258e-13 & 2.3288e-11 
\end{array}
}
\right)
$$
\textbf{(ii) show that it achieves the predicted closed-loop $H_2$ gain.}\\
(Returned from Matlab SS System Analysis) $H_2$ gain = 1.9340\\

Therefore the controller approx achieves the predicted gain since the following terms are approximately equal:\\
$H_2(LMI)$ = $H_2(controller)$;1.934=1.934
\end{tcolorbox}


\section{8}
\subsection{Problem 8.a}
\begin{tcolorbox}
\textbf{SOLUTION:}\\

The mixed $H_2-H_\infty$ LMI aims to optimize both norms.  
\begin{center}
$||S(K,P)_{H_2}<\gamma_1$ and $||S(K,P)_{H_{\infty}}<\gamma_2$
\end{center}
To reformulate the problem to find
\begin{center}
$min_K||S(K,P)||^2_{H_2}+||S(K,P)||^2_{H_\infty}$\\
$||S(K,P)^2_{H_2}<\gamma_1^2$ and $||S(K,P)^2_{H_{\infty}}<\gamma_2^2$
\end{center}
we must address the difference between the $H_2$ optimization problem ($||S(K,P)||^2_{H_2}<\gamma_1$) and the $H_\infty$ optimization problem ($||S(K,P)||_{H_\infty}<\gamma_2$) by changing the $H_\infty$ optimization problem to ($||S(K,P)||^2_{H_\infty}<\gamma^2_2$) where $\beta_2=\gamma^2_2$ giving our minimization problem
%$$min_K(weight_1*\gamma_1+weight_2*\gamma^2_2)$$
$$min_K(weight_1*\gamma_1+weight_2*\beta_2)$$


\textbf{$H_2$ Constraints}\\
Constraint for the $H_2$ norm that ensures norm is less than $\gamma_1$
$$\begin{bmatrix}
(A*Y1+Y1*A'+B2*Cn+Cn'*B2')&  *^T& *^T\\
(A'+An+(B2*Dn*C2)')& (X1*A+A'*X1+Bn*C2+C2'*Bn')&  *^T\\
((B1+B2*Dn*D21)')& ((X1*B1+Bn*D21)')& (-I)
\end{bmatrix}<0$$

$$\begin{bmatrix}
(Y1)&  I&  *^T\\
I&  (X1)&  *^T\\
(C1*Y1+D12*Cn)&	(C1+D12*Dn*C2)&	(Z)
\end{bmatrix}>0
$$
$$(D11+D12*Dn*D21)=0$$
$$trace(Z)<\gamma_1^2$$
Since the constraint $trace(Z) < \gamma_1^2$ is not linear we must make variable substitutions for $\beta_1 = \gamma_1^2$\\
$$trace(Z)<\beta_1$$
and we make the substitution in our original optimization problem
$$min_K(weight_1*\beta_1+weight_2*\beta_2)$$

\textbf{$H_\infty$ Constraints}\\
Constraint for the $H_\infty$ norm that ensures norm is less than $\gamma_2$\\
$$\scalemath{0.7}{\begin{bmatrix}
(A*Y1+Y1*A'+B2*Cn+Cn'*B2')&  *^T&	*^T&	*^T\\
(A'+An+(B2*Dn*C2)')&	(X1*A+A'*X1+Bn*C2+C2'*Bn')&	*^T&	*^T\\
((B1+B2*Dn*D21)')&	(X1*B1+Bn*D21)'&	(-\beta_2*I)&	*^T\\
(C1*Y1+D12*Cn)&	(C1+D12*Dn*C2)&	(D11+D12*Dn*D21)&	(-\beta_2*I)
\end{bmatrix}}<0
$$

However, we can only optimize for one variable at a time and therefore must create a new variable to represent a weighted combination of the two: 
$$\beta=weight_1*\gamma^3_1+weight_2*\gamma^2_2=weight_1*\beta_1+weight_2*\beta_2$$
$$min_K(\beta)$$

Once the optimization for the system is ran in Yalmip, we will be left with values for $\beta_1$ and $\beta_2$ which we can then recover the values for $\gamma_1$ and $\gamma_2$ by taking the square root.

\end{tcolorbox}



\subsection{Problem 8.b}
\begin{tcolorbox}
\textbf{SOLUTION:}\\
Using the constraints from part b, the equally waited optimal-feedback problem was solved by optimizing for:
$$\beta=weight_1*\beta_1+weight_2*\beta_2=\beta_1+\beta_2$$

\textbf{(i) Matlab Result:}\\
$min_K||S(K,P)||^2_{H_2}<2.0988$
Minimized $H_2$ Gain = 2.0988
$+||S(K,P)||^2_{H_\infty} <1.8048$
Minimized $H_\infty$ Gain = 1.8048
\end{tcolorbox}
\subsection{Problem 8.c}
\begin{tcolorbox}
\textbf{SOLUTION:}\\
\textbf{(i) Construct the corresponding controller}\\
\textit{Note: Controller constructed according the same process in 7.b}
$$K=1e8* \left(\scalemath{0.9}{\begin{array}{ccccccccc} 
0.1718 & -0.0992 & -0.3069 & 0.0356 & -0.1671 & -0.2842 & -0.1334 & -0.0206 & -0.0713\\
-1.1208 & 0.8249 & 2.7836 & 0.2750 & 1.0416 & 1.9349 & 1.0135 & 0.2476 & 0.5326\\
0.3162 & 0.2165 & 0.7141 & 0.0314 & 0.2982 & 0.5384 & 0.2728 & 0.0596 & 0.1441\\
0.6676 & -0.9288 & -3.5851 & -1.4153 & -0.5001 & -1.3508 & -0.9573 & -0.4263 & -0.4840\\
0.8452 & -0.6288 & -2.1290 & -0.2268 & -0.7836 & -1.4621 & -0.7698 & -0.1911 & -0.4042\\
1.4425 & -1.0623 & -3.5856 & -0.3559 & -1.3403 & -2.4905 & -1.3050 & -0.3191 & -0.6858\\
0 & 0 & 0 & 0 & 0 & 0 & 0 & 0 & 0\\
0 & 0 & 0 & 0 & 0 & 0 & 0 & 0 & 0 \end{array}}\right)$$

\textbf{(ii) Determine and compare the resulting H2 gain to the gains predicted by the LMI}\\
(Returned from Matlab SS System Analysis) $H_2$ gain = 1.9512\\
    
Therefore the controller approx achieves the predicted gain since the following terms are approximately equal:\\
   $H2(LMI) = H2(controller)$;  2.0988$>$1.9512\\
   
   
\textbf{(iii) Determine and compare the resulting $H_inf$ gain to the gains predicted by the LMI}\\
(Returned from Matlab SS System Analysis)$H_inf$ gain =  1.5708\\

Therefore the controller approx achieves the predicted gain since the following terms are approximately equal:\\
   $H_\infty(LMI)$ = $Hinf_(controller)$; 1.8048$>$1.5708
   
\end{tcolorbox}
\subsection{Problem 8.d}
\begin{tcolorbox}
\textbf{SOLUTION:}\\
\begin{center}
Individual Optimization Gains $\leq$ Mixed Optimization Gains
\end{center}
$$Individal_{H_2}=1.934 \leq 2.0988 = Mixed_{H_2}$$
$$Individal_{H_\infty} = 1.1070 \leq 1.8048 = Mixed_{H_\infty}$$

This observation makes sense since we cannot truly optimize for two variables at the same time because optima for one gain may not produce a controller that is optimal for the other.  In a sense, a compromise is made between the the two objective functions ($\beta=weight_1*\beta_1+weight_2*\beta_2$) so that both are bounded but neither are likely to be optimal.  However, there is a chance that the controller that is returned from the optimization is optimal for both (hence the $\leq$ instead of $<$) but this is a product of circumstance.


\end{tcolorbox}

\newpage
\section{Matlab Main Function: Problem 6}
\lstinputlisting{MAE598_HW3_P6_Smith.m}

\newpage
\section{Matlab Main Function: Problem 7}
\lstinputlisting{MAE598_HW3_P8_Smith.m}

\newpage
\section{Matlab Main Function: Problem 8}
\lstinputlisting{MAE598_HW3_P8_Smith.m}

\end{document}

